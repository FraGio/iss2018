\documentclass{../libs/llncs}
%%%%%%%%%%%%%%%%%%%%%%%%%%%%%%%%%%%%%%%%%%%%%%%%%%%%%%%%%%%
%% package sillabazione italiana e uso lettere accentate
\usepackage[latin1]{inputenc}
\usepackage[english]{babel}
\usepackage[T1]{fontenc}
%%%%%%%%%%%%%%%%%%%%%%%%%%%%%%%%%%%%%%%%%%%%%%%%%%%%%%%%%%%%%

\usepackage{url}
\usepackage{xspace}

\makeatletter
%%%%%%%%%%%%%%%%%%%%%%%%%%%%%% User specified LaTeX commands.
\usepackage{../libs/manifest}

\makeatother


%%%%%%%
 \newif\ifpdf
 \ifx\pdfoutput\undefined
 \pdffalse % we are not running PDFLaTeX
 \else
 \pdfoutput=1 % we are running PDFLaTeX
 \pdftrue
 \fi
%%%%%%%
 \ifpdf
 \usepackage[pdftex]{graphicx}
 \else
 \usepackage{graphicx}
 \fi
%%%%%%%%%%%%%%%
 \ifpdf
 \DeclareGraphicsExtensions{.pdf, .jpg, .tif}
 \else
 \DeclareGraphicsExtensions{.eps, .jpg}
 \fi
%%%%%%%%%%%%%%%

\newcommand{\java}{\textsf{Java}}
\newcommand{\contact}{\emph{Contact}}
\newcommand{\corecl}{\texttt{corecl}}
\newcommand{\medcl}{\texttt{medcl}}
\newcommand{\msgcl}{\texttt{msgcl}}
\newcommand{\android}{\texttt{Android}}
\newcommand{\dsl}{\texttt{DSL}}
\newcommand{\jazz}{\texttt{Jazz}}
\newcommand{\rtc}{\texttt{RTC}}
\newcommand{\ide}{\texttt{Contact-ide}}
\newcommand{\xtext}{\texttt{XText}}
\newcommand{\xpand}{\texttt{Xpand}}
\newcommand{\xtend}{\texttt{Xtend}}
\newcommand{\pojo}{\texttt{POJO}}
\newcommand{\junit}{\texttt{JUnit}}

\newcommand{\action}[1]{\texttt{#1}\xspace}
\newcommand{\code}[1]{{\small{\texttt{#1}}}\xspace}
\newcommand{\codescript}[1]{{\scriptsize{\texttt{#1}}}\xspace}

% Cross-referencing
\newcommand{\labelsec}[1]{\label{sec:#1}}
\newcommand{\xs}[1]{\sectionname~\ref{sec:#1}}
\newcommand{\xsp}[1]{\sectionname~\ref{sec:#1} \onpagename~\pageref{sec:#1}}
\newcommand{\labelssec}[1]{\label{ssec:#1}}
\newcommand{\xss}[1]{\subsectionname~\ref{ssec:#1}}
\newcommand{\xssp}[1]{\subsectionname~\ref{ssec:#1} \onpagename~\pageref{ssec:#1}}
\newcommand{\labelsssec}[1]{\label{sssec:#1}}
\newcommand{\xsss}[1]{\subsectionname~\ref{sssec:#1}}
\newcommand{\xsssp}[1]{\subsectionname~\ref{sssec:#1} \onpagename~\pageref{sssec:#1}}
\newcommand{\labelfig}[1]{\label{fig:#1}}
\newcommand{\xf}[1]{\figurename~\ref{fig:#1}}
\newcommand{\xfp}[1]{\figurename~\ref{fig:#1} \onpagename~\pageref{fig:#1}}
\newcommand{\labeltab}[1]{\label{tab:#1}}
\newcommand{\xt}[1]{\tablename~\ref{tab:#1}}
\newcommand{\xtp}[1]{\tablename~\ref{tab:#1} \onpagename~\pageref{tab:#1}}
% Category Names
\newcommand{\sectionname}{Section}
\newcommand{\subsectionname}{Subsection}
\newcommand{\sectionsname}{Sections}
\newcommand{\subsectionsname}{Subsections}
\newcommand{\secname}{\sectionname}
\newcommand{\ssecname}{\subsectionname}
\newcommand{\secsname}{\sectionsname}
\newcommand{\ssecsname}{\subsectionsname}
\newcommand{\onpagename}{on page}

%AUTH
\newcommand{\xauthA}{Andrea Torchi }
%\newcommand{\xauthB}{Francesco Giovanelli}
%\newcommand{\xauthC}{Giuseppe Tempesta }

\newcommand{\xfaculty}{II Faculty of Engineering}
\newcommand{\xunibo}{Alma Mater Studiorum -- University of Bologna}
\newcommand{\xaddrBO}{viale Risorgimento 2}
\newcommand{\xaddrCE}{via Venezia 52}
\newcommand{\xcityBO}{40136 Bologna, Italy}
\newcommand{\xcityCE}{47023 Cesena, Italy}

%
% Comments
%
%%% \newcommand{\todo}[1]{\bf{TODO:}\emph{#1}}


\begin{document}

\title{Software Engineering\\
 process report template}

%%% \author{\xauthA \and \xauthB}
\author{\xauthA}

\institute{%
	%%%  \xunibo\\\xaddrCE, \xcityCE\\\email{\{nameA.studentA, nameB.studentB\}@studio.unibo.it}
	\xunibo\\\xaddrCE, \xcityCE\\\email\ andrea.torchi@studio.unibo.it
%	\xunibo\\\xaddrCE, \xcityCE\\\email\ francesco.giovanelli@studio.unibo.it
%	\xunibo\\\xaddrCE, \xcityCE\\\email\ giuseppe.tempesta2@studio.unibo.it
}

\maketitle

%% \begin{abstract}
%% \footnotesize
%%This a Latex template to be used for the reports of Software Engineering.
%%\keywords{Software engineering, managed software development, reports, ....}
%%\end{abstract}

%%% \sloppy

%===========================================================================
\section{Introduction}
\labelsec{intro}
Ai problemi ci si pu� approcciare in modo olistico (top-down), o riduzionistico (bottom-up). 
Cosa intendiamo?\\
TOP-DOWN, cio� procedendo dall'alto verso il basso, quindi partendo da specifiche di alto livello, analizzando i sottosistemi ed arrivando ad una soluzione; questo approccio � in linea con una visione OLISTICA, ovvero che non si concentra sulle singole parti, ma ha una visione "pi� d'insieme" o di alto livello. \\
BOTTOM-UP, cio� procedendo dal basso verso l'alto, quindi partendo pi� da componenti, per poi passare alla loro sintesi. Questo approccio � in linea con una visione RIDUZIONISTA, opposta alla precedente, la quale mira a porre l'attenzione sulle singole parti di un certo sistema, piuttosto che sull'insieme intero.\\
%===========================================================================

%===========================================================================
\section{Vision}
\labelsec{Vision}
La "Visione" � una frase che fa capire come ci si approcci alle cose, ovvero come affrontare problemi.
Lo scopo dell'analisi dei requisiti � quello di capire il problema, analizzando i requisiti ed evidenziando aspetti problematici, successivamente, produrre (in modo formale) uno o pi� modelli che rappresentino il sistema. \\
Possibile affrontare un problema partendo da zero? Senza alcuna ipotesi? Molto difficile, quasi impensabile!
Partire dal foglio bianco significa non avere alcuna ipotesi tecnologica (come se si partisse assolutamente privi di informazioni, non ho nulla su cui orientarmi). \\
La questione che adesso ci si pone � l'affrontare un problema partendo da ipotesi tecnologiche e poi arrivando ad una soluzione utilizzando gli strumenti che si conoscono (approccio bottom-up). L'approccio giusto, in ogni caso, � decidere il pi� tardi possibile quale tecnologia utilizzare, poich� si vuole trovare quella giusta(la tecnologia), al momento giusto. Non si costruisce in funzione della "scatola lego" (cio� dei pezzi che hai a disposizione), ma si analizza quello che si vuol fare e, solo dopo si cerca la "scatola lego" pi� opportuna a per quello che si vuol realizzare. \\
La visione adottata in questo caso �:
"Dalle tecnologie alla analisi e al progetto logico e ritorno alle tecnologie." 
Cosa significa? \\
Dopo aver capito che comunque si deve partire dalle tecnologie (e quindi non si pu� partire da zero, senza alcuna ipotesi tecnologica), si fanno delle ipotesi su cosa fare. Poi ci si occupa di analisi dei requisiti. Successivamente si ritorna alle tecnologie per poter dire se si ha un "abstraction-gap". Quando si pu� dire se si � in presenza di una cosa del genere? Se nella business logic per ogni byte se ne devono scrivere 100 per l'infrastruttura, significa che c'� un abstraction gap enorme. Quindi la tecnologia che sto utilizzando � insufficiente, o meglio, inappropriata per il mio problema. \\
Si parla di "tecnology-lock" se l'applicazione � strettamente contaminata dalla tecnologia, ovviamente potrebbe rappresentare un problema. Questo accade quando la scelta della tecnologia viene fatta prima rispetto le scelte di analisi/progetto; si contamina/rende strettamente dipendente il sistema finale dalla tecnologia. \\
Altra cosa importante da tenere a mente �: "non c'� codice senza progetto, non c'� progetto senza analisi e non c'� analisi senza requisiti". 
%===========================================================================

%===========================================================================
\section{Goals}
\labelsec{Goals}
%===========================================================================

%===========================================================================
\section{Requirements}
\labelsec{Requirements}
Our current robot system can be controlled in remote way in different ways: \\\\
1. by an human user using the web interface provided by the robot executor;  \\
2. by an human user using the web interface provided by the real robot; \\
3. by a machine that sends command messages to the robot or that emits command events that can be understood by the robot. \\\\
These multiple possibilities are very useful during software development and testing, but are source of confusion
when we want to allow the usage of a robot as a resource conceptually owned by a single user and controlled by
means of a single, certified interface.\\
Thus, we want extend our system with a set of new requirements: \\\\
1. The physical robot must expose in a visible way a Led and: \\
- the Led must be on when the robot is engaged by an user (human or machine); \\
- the Led must be off when the robot is available for booking. \\
2. the robot system does not expose any public available usage interface;
3. in order to use the robot, an user must first of all send 'to the system' a booking request. The system
must return an answer including an access token if the robot is available. If the answer is negative, (robot
already engaged) and the request includes a 'notify-me flag', the system must notify the user when to robot
becomes again available; \\
4. the user that receives the access token must send within a given acquisition-deadline (e.g. 30 sec)
the request for a robot-driving command interface, by appending to the request the access token. If the
acquisition-deadline expires, the robot returns in its 'available state'; \\
5. the user can use the robot-driving command interface at most for a prefixed usage-duration time; \\
6. the user can explicitly release the robot resource by sending a booking release message; \\
7. if many users attempt to book the robot resource 'at the same time', the system could operate in two different
ways: \\
(a) by selecting the first emitted request; \\
(b) by selecting the first received request; \\
%===========================================================================

 
%===========================================================================
\section{Requirement analysis}
\labelsec{ReqAnalysis}
%===========================================================================
\subsection{Use cases}
\labelssec{UseCases}

\subsection{Scenarios}
\labelssec{Scenarios}

\subsection{(Domain)model}

\subsection{Test plan}

%===========================================================================
\section{Problem analysis}
\labelsec{ProblemAnalysis}
%===========================================================================
\subsection{Logic architecture}
\subsection{Abstraction gap}
\subsection{Risk analysis}

%===========================================================================
\section{Work plan}
\labelsec{wplan}
%===========================================================================

%===========================================================================
\section{Project}
\labelsec{Project}
%===========================================================================

\subsection{Structure}
\subsection{Interaction}
\subsection{Behavior}

%===========================================================================
\section{Implementation}
\labelsec{Implementation}
%===========================================================================

%===========================================================================
\section{Testing}
\labelsec{testing}
%===========================================================================

%===========================================================================
\section{Deployment}
\labelsec{Deployment}
%===========================================================================

%===========================================================================
\section{Maintenance}
\labelsec{Maintenance}
%===========================================================================
\newpage
See \cite{natMol09} until page 11 (\texttt{CMM}) and pages 96-105.

%===========================================================================
\section{Information about the author}
\labelsec{Author}
%===========================================================================

\vskip.5cm
%%% \begin{figure}
\begin{tabular}{ | c |  }
\hline
  % after \\: \hline or \cline{col1-col2} \cline{col3-col4} ...
  Photo of the author 
  \\
\hline
   %\includegraphics[scale = 0.3]{}
  \\
\hline
\end{tabular}


%%% \begin{itemize}
%%% \item Titolo di studio:\\ \\
%%% \item Interessi particolari:\\ \\
%%% \item Ha sostenuto fino ad oggi il seguente numero di esami:\\ \\
%%% \item Deve ancora sostenere i seguenti esami del I anno:\\ \\
%%% \item Prevede di svolgere un tirocinio presso:\\ \\
%%% \item Prevede di laurearsi nella sessione:\\ \\
%%% \item Intende proseguire gli studi per conseguire: \\  \\  \\
%%%   	presso la sede universitaria di: \\ \\
%%% \item Intende entrare subito nel mondo del lavoro presso : \\ \\
%%% \end{itemize}

 
\appendix


\bibliographystyle{abbrv}
\bibliography{biblio}

\end{document}












